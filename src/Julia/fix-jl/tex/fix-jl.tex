

\documentclass{article}
\usepackage[utf8]{inputenc}
\usepackage{setspace}
\usepackage{ mathrsfs }
\usepackage{graphicx}
\usepackage{amssymb} %maths
\usepackage{amsmath} %maths
\usepackage[margin=0.2in]{geometry}
\usepackage{graphicx}
\usepackage{ulem}
\setlength{\parindent}{0pt}
\setlength{\parskip}{10pt}
\usepackage{hyperref}
\usepackage[autostyle]{csquotes}

\usepackage{cancel}
\renewcommand{\i}{\textit}
\renewcommand{\b}{\textbf}
\newcommand{\q}{\enquote}
%\vskip1.0in





\begin{document}

{\setstretch{0.0}{
Fix [ Julia Version ]

\b{Fix} is a simplification of the \b{Prefix} (and also of the \b{Revolver}) crypto-system.  The innovation is making the set of names of the states itself an instantaneous code. Then each state also writes a subset of other state names, depending on the single input symbol of the plaintext. This means that each state (its name part of the larger prefix code) also writes a prefix code (a subset of state names), and this is how the function stays invertible. 


}}
\end{document}
