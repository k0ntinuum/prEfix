

\documentclass{article}
\usepackage[utf8]{inputenc}
\usepackage{setspace}
\usepackage{ mathrsfs }
\usepackage{amssymb} %maths
\usepackage{amsmath} %maths
\usepackage[margin=0.2in]{geometry}
\usepackage{graphicx}
\usepackage{ulem}
\setlength{\parindent}{0pt}
\setlength{\parskip}{10pt}
\usepackage{hyperref}
\usepackage[autostyle]{csquotes}

\usepackage{cancel}
\renewcommand{\i}{\textit}
\renewcommand{\b}{\textbf}
\newcommand{\q}{\enquote}
%\vskip1.0in



\begin{document}

\begin{huge}



{\setstretch{0.0}{

\b{Refix} is a variant of Prefix. Instead of using modes which each contain their own prefix code, Refix uses mini-states which either consume or fail to consume a fixed pattern in the current symbol(s) of plain text. If the ministate finds its pattern, it consumes the this input pattern and writes out a success output pattern, jumping forward a fixed number of ministates to its next ministate. If the ministate fails to find its fixed pattern, it writes a failure output pattern. It then \q{falls through} to the next ministate in the key. 
Note that the key is just a (circular) array of such ministates. The jumps on success are rotated independently of their associated patterns so that the key evolves as a function of the plaintext.
 


}}
\end{huge}
\end{document}
