

\documentclass{article}
\usepackage[utf8]{inputenc}
\usepackage{setspace}
\usepackage{ mathrsfs }
\usepackage{amssymb} %maths
\usepackage{amsmath} %maths
\usepackage[margin=0.2in]{geometry}
\usepackage{graphicx}
\usepackage{ulem}
\setlength{\parindent}{0pt}
\setlength{\parskip}{10pt}
\usepackage{hyperref}
\usepackage[autostyle]{csquotes}

\usepackage{cancel}
\renewcommand{\i}{\textit}
\renewcommand{\b}{\textbf}
\newcommand{\q}{\enquote}
%\vskip1.0in



\begin{document}

\begin{huge}



{\setstretch{0.0}{
\b{PRE}

Pre or Prefix is a family of symmetric cryptosystems that share the feature of states that write prefix or instantaneous codes. In other words, a Prefix machine outputs a variable number of symbols. In some versions the machine also consumes a variable number of symbols. 

I've experimented with this idea in several languages. The trickier part to write is not the encoder and decoder but rather the part that generates a random key, something that is pretty easy for humans to do, but also tedious. I hope to further develop this project and add more information here. The simplest version (most compact) version is p-js, a JavaScript version without a key generator. Even in this symbol version the key evolves as a function of the particular plaintext being encoded.
}}
\end{huge}
\end{document}
